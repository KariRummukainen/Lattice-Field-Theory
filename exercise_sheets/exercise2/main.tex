\documentclass{article}
\usepackage{graphicx} 
\usepackage{enumitem} 
\usepackage{listings}


\title{Lattice Field Theory - Exercise Session 25.1-01.2}

\begin{document}

\maketitle

This exercise session will be centered around autocorrelation time and behaviour of Ising 2D with different algorithms. You will need the Metropolis-Hastings version of the Ising 2D code used in the previous exercise. Feel free to use your own, or the uploaded Python solution of exercise 1 (on github). For this exercise session, we will operate with no external field $\bar{h}=0$ and $\bar{J}=1$. 

\begin{enumerate}

\item  Let us now implement the Cluster Wolff algorihm and plot the magnetization. Choose a warm up of 1000 steps. Add 1000 computation steps. For this algorithm, begin by,
\begin{enumerate}
\item 1. Choose site x,y at random
\item Precompute probability of adding site to cluster $P_{add}= 1- exp(-2J/T)$
\item Initialize two lists of coordinates, one named Cluster another Pocket. One will contain all coordinates belonging to a cluster of spins, the other of immediate neighbours to which one should jump to, in order to grow the cluster. Something like Cluster = [[x,y]] should do the trick.
\item Now this is the hard part, and for this reason I've elected to provide some pseudocode:

\begin{lstlisting}

    while Pocket is not empty:
        Pocket_new = []
        for i,j coordinate pairs in Pocket:
	    # find all the neighbours, 
	    #remember to wrap around boundaries
            ip1 = (i+1) % lattice_size
            im1 = (i-1+lattice_size) %lattice_size
            jp1 = (j+1) % lattice_size
            jm1 = (j-1+lattice_size) %lattice_size
            nbr = [[ip1,j], 
                   [im1,j], 
                   [i,jp1], 
                   [i,jm1]]
	    #Now we go over all neighbours of current spin
            for l in nbr:
                if spin[l] == spin[current coordinate in Pocket] 
			if l not in Cluster:
                    		if random number < P_add:
					Add spin[l] to Pocket_new 
					Add spin[l] to Cluster
	# Remember to reset Pocket, 
	# we need to keep growing the cluster
	# so we must jump to new spins
        Pocket = Pocket_new 
\end{lstlisting}

\item Flip all spins in cluster - done!

Be very careful to get all of the indents right! After all of this a single update sweep of Cluster Wolff is coded. Show me what the magnetization $|M|$ looks like in the range $T\in[1,5]$.
\end{enumerate} 

\item Let us implement the autocorrelation function of the magnetization observable. Reminder the autocorrelation function is given by,

\begin{equation}
	C(t) = \frac{\frac{1}{N-t} \sum_{i=1}^{N-t} \bigg[ X_{i} X_{i+t} \bigg] -\langle X \rangle_1 \langle X \rangle_2}{\langle X^2 \rangle - \langle X \rangle ^2}
\end{equation}

where,

\begin{equation}
\langle X \rangle_1 =  \frac{1}{N-t} \sum_{i=1}^{N-t} X_i ,
\end{equation}
\begin{equation}
\langle X \rangle_2 =  \frac{1}{N-t} \sum_{i=t}^{N} X_{i},
\end{equation}

Plot the autocorrelation function normalized with the first step, $C(t)/C(0)$ as a function of steps for three different temperatures, $T\in[1,5]$. Trick: compute C(t) with $t\in [0,t_{max}]$, where $t_{max}=70$. This is necessary to also remove the decorrelated noise at large $t$. Has the advantageof being computationally easier. How does this compare with Metropolis, is the exponential fall-off faster or slower?

 
\end{enumerate}

\end{document}
