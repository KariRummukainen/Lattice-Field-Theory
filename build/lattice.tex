%% LyX 2.2.4 created this file.  For more info, see http://www.lyx.org/.
%% Do not edit unless you really know what you are doing.
\documentclass[english]{article}
\usepackage[T1]{fontenc}
\usepackage[latin9]{inputenc}
\pagestyle{plain}
\setcounter{tocdepth}{2}
\usepackage{esint}

\makeatletter
%%%%%%%%%%%%%%%%%%%%%%%%%%%%%% User specified LaTeX commands.
\usepackage{slashed}
% Added by lyx2lyx
\renewcommand{\textendash}{--}
\renewcommand{\textemdash}{---}

\makeatother

\usepackage{babel}
\begin{document}

\title{Lattice Field Theory}

\maketitle
\tableofcontents{}

\section{Motivation}

\section{Learning Objectives}
\begin{itemize}
\item Can implement a Montecarlo simulation of a quantum field theory in
discrete space
\item Can analyse the output of the calculation and describe it's connection
to the theory
\item Recognizes the most common discrete representations of field theories
and is able to find information on them
\item Recognizes observables in the discrete representations and is able
to find information on them
\item Can apply perturbation theory in a discrete space
\end{itemize}

\section{Spin models}

\section{Thermodynamics of lattice models}

\section{The lattice regularization}

\section{Thermal lattice field theory}

\section{Scalar fields}

\section{Gauge fields}

\section{Fermions}

\section{Lattice perturbation theory}

\begin{equation}
<O>=\frac{\int_{U}Z(U)O}{\int_{U}Z(U)}
\end{equation}

\end{document}
